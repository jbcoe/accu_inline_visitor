%\documentclass[10pt,a4paper,twocolumn]{article}
\documentclass[10pt,a4paper]{article}

% Use packages
\usepackage{newcent}
\usepackage{amssymb}
\usepackage{amsmath}
\usepackage{graphicx}
\usepackage{mathrsfs}
\usepackage{color}

% Set lengths
\setlength{\parindent}{0pt}
\setlength{\parskip}{10pt}
\addtolength{\columnsep}{0.05in}
\addtolength{\textwidth}{1in}
\addtolength{\oddsidemargin}{-0.5in}
\addtolength{\topmargin}{-1in}
\addtolength{\textheight}{1.5in}

% Little hypenation, left aligned
\hyphenpenalty=10000

\makeatother

\renewcommand\section[1]{
    \begin{minipage}[c]{0.94\linewidth}
    \large \raggedright \sffamily \textbf{#1}
    \end{minipage}
}

\renewcommand\subsection[1]{
    \begin{minipage}[c]{0.94\linewidth}
    \raggedright \sffamily \textbf{#1}
    \end{minipage}
}

\newcommand\mycode[1]{{\small\texttt{#1}}}

\makeatletter

\title{Defining Visitors Inline in Modern C++}
\author{Robert W. Mill \and Jonathan B. Coe}

\begin{document} 
\maketitle

The visitor pattern can be useful when type-specific handling is required and
tight coupling of type-handling logic and handled types is either an acceptable
cost or desirable in its own right. We've found that selective application of
the visitor pattern adds strong compile-time safety, as the handling of new types
needs explicit consideration in every context where type-specific handling
occurs. The visitor pattern presents an inversion of control that can feel
unnatural and often requires introduction of considerable non-local boilerplate
code. We've found that this slows adoption of the visitor pattern especially
among engineers and scientists who traditionally write their type-handling
logic inline. Here we present a solution for defining visitors inline.

\section{The Problem}

In object-oriented programming, we may need to perform a function on an object
of polymorphic type, such that the behaviour of the function is specific to the
derived type. Suppose that for the abstract base class \mycode{Polygon} we
derive the concrete classes \mycode{Triangle} and \mycode{Square}. The free
function \mycode{CountSides}, returns the number of sides in the polygon,
\mycode{p}.

{\small\begin{verbatim}struct Polygon {};

struct Triangle : Polygon 
{ 
  // members 
}

struct Square : Polygon 
{ 
  // members 
}

int CountSides(Polygon& p)
{ 
  // implementation 
}\end{verbatim}}

\mycode{CountSides} will need the derived type of the polygon \mycode{p} to
compute its result, which is problematic, because its argument is conveyed by a
reference of the base class type, \mycode{Polygon}.

\section{Visitor pattern}

The \emph{visitor} design pattern offers a mechanism for type-specific handling
using virtual dispatch [REF: Gang of four]. In the words of Scott Meyers: 
"Visitor lets you define a new operation without changing the classes of the
elements on which it operates" [REF: Scott]. The pattern uses the \mycode{this}
pointer inside the class to identify the derived type. Each derived object must
\emph{accept} a visitor interface which provides a list of \mycode{Visit}
members with a single argument overloaded on various derived types.

To continue our illustration, the \mycode{PolygonVisitor} is able to
\emph{visit} \mycode{Triangle}s and \mycode{Square}s, and all these polygons
must be able to \emph{accept} a \mycode{PolygonVisitor}.

{\small\begin{verbatim} 
struct Triangle; 
struct Square;

struct PolygonVisitor 
{ 
  virtual ~PolygonVisitor() {}

  virtual void visit(Triangle& tr) = 0; 
  virtual void visit(Square& sq) = 0;
};

struct Polygon 
{ 
  virtual void accept(PolygonVisitor& v) = 0; 
};\end{verbatim}}

\mycode{Square}s and \mycode{Triangle}s accept the visitor as follows. Observe
that the \mycode{this} pointer is used to select the appropriate overloaded
function in the visitor interface.

{\small\begin{verbatim} 
struct Triangle : Polygon 
{ 
  void accept(PolygonVisitor& v) override 
  { 
    v.Visit(*this); 
  } 
};

struct Square : Polygon 
{ 
  void accept(PolygonVisitor& v) override 
  {
    v.Visit(*this); 
  } 
};\end{verbatim}}

A visitor object, \mycode{SideCounter}, which counts the number of sides of a
polygon and stores the result, is implemented and used as follows.

{\small\begin{verbatim} 
struct SideCounter : PolygonVisitor 
{ 
  void visit(Square& sq) override 
  { 
    m_sides = 4; 
  }
  
  void visit(Triangle& tr) override 
  { 
    m_sides = 3; 
  }
  
  int m_sides = 0; 
};

int CountSides(Polygon& p) 
{ 
  SideCounter sideCounter; 
  p.Accept(sideCounter);
  return sideCounter.m_sides; 
}\end{verbatim}}

%[JBCOE FORCED FORMATTING]
\newpage

\section{Inline visitor pattern}

One potential drawback of the visitor pattern is that it requires the creation
of a new visitor object type for each algorithm that operates on the derived
type. In some cases, the class created will not be reused and, much like a
lambda, it would be more convenient to write the visitor clauses inline. The
listing below shows how this can be accomplished in a form that resembles a
\mycode{switch} statement.

{\small\begin{verbatim} 
int CountSides(Polygon& p) 
{ 
  int sides = 0;
  
  auto v = begin_visitor<PolygonVisitor> 
    .on<Triangle>([&sides](Triangle& tr) 
    {
      sides = 3; 
    }) 
    .on<Square>([&sides](Square& sq) 
    { 
      sides = 4; 
    }) 
    .end_visitor();
  
  p.Accept(v); 
  return sides; 
} \end{verbatim}}

In Listing~1 we demonstrate generic code that permits the
\mycode{begin\_visitor} ... \mycode{end\_visitor} construction to be used with
any visitor base. The initial \mycode{start\_visitor} call instantiates a class
which defines an inner object inheriting from the visitor interface; each
subsequent call of the \mycode{on} function instantiates a class whose inner
class inherits from the previous inner class implementing an additional
\mycode{Visit} function.  Finally the \mycode{end\_visitor} call returns an
instance of the the inner visitor class.

\subsection{Listing 1}

{\small
\begin{verbatim}
template <typename T,
          typename F,
          typename BaseInner,
          typename ArgsT>
struct ComposeVisitor
{
  struct Inner : public BaseInner
  {
    using BaseInner::Visit;
  
    Inner(ArgsT&& args) :
      BaseInner(move(args.second)),
      m_f(move(args.first))
    {
    }
  
    void Visit(T& t) final override
    {
      m_f(t);
    }
  
  private:
    F m_f;
  };
  
  ComposeVisitor(ArgsT&& args) :
    m_args(move(args))
  {
  }

  template <typename Tadd,
            typename Fadd>
  ComposeVisitor<
    Tadd,
    Fadd,
    Inner,
    pair<Fadd, ArgsT>> on(Fadd&& f)
  {
    return ComposeVisitor
      Tadd,
      Fadd,
      Inner,
      pair<Fadd, ArgsT>>(
        make_pair(
          move(f),
          move(m_args)));
  }
  
  Inner end_visitor()
  {
    return Inner(move(m_args));
  }
  
  ArgsT m_args;
};

template <typename TVisitorBase>
struct EmptyVisitor
{
  struct Inner : public TVisitorBase
  {
    using TVisitorBase::Visit;
  
    Inner(nullptr_t) {}
  };
  
  template <typename Tadd, typename Fadd>
  ComposeVisitor<
    Tadd,
    Fadd,
    Inner,
    pair<Fadd, nullptr_t>> on(Fadd&& f)
  {
    return ComposeVisitor<
      Tadd,
      Fadd,
      Inner,
      pair<Fadd, nullptr_t>>(
        make_pair(
          move(f),
          nullptr));
  }
};

template <typename TVisitorBase>
EmptyVisitor<TVisitorBase> begin_visitor()
{
  return EmptyVisitor<TVisitorBase>();
}
\end{verbatim}}

The consistency between the list of types used with \mycode{on} and those in
the visitor base is verified at compilation time. Since the \mycode{override}
qualifier is specified on the \mycode{Visit} member function, it is not
possible to add a superfluous \mycode{Visit} which does not correspond to a
type overload in the visitor base. Similarly, because the \mycode{final}
qualifier is specified on the \mycode{Visit} member function it is not possible
to define a \mycode{Visit} member function more than once. 

That inline visitors cannot be constructed when clauses are missing may also be
considered desirable in some contexts. For instance, if a new type
\mycode{Hexagon} is derived from \mycode{Polygon}, then the code base will
compile only when appropriate \mycode{Visit} functions been introduced to
handle it. In large code bases, this may serve maintainability. If it is deemed
that a visitor clause should have some default behaviour (e.g., no operation),
a concrete visitor base can be passed into \mycode{start\_visitor}.


\section{Performance}

With optimizations turned on MSVC 2013, GCC 4.9.1 and Clang 3.4.2 compile the
inline visitor without introducing any cost. GCC and Clang produce identical
assembly code in the case when a visitor class is explicitly written out. MSVC
produces different assembly code for the inline visitor and explicit visitor
class; the inline visitor has been measured to run marginally faster.


\section{Other Visitor}

Loki's Acyclic Visitor [REF: Acyclic Visitor, Loki] removes compile-time coupling
from visiting and visited classes but at the cost of introducing dynamic casts
and run-time detection of unhandled types. When run-time performance and
compile-time detection of unhandled types are favoured over shorter
compile-times then we would recommend use of the inline visitor.

The inline visitor does not have the flexibility of the Cooperative Visitor
[REF: Cooperative visitor], which allows different method names and return
types, but as it is intended to be lightweight this flexibility is not needed:
the visit functions are not explicitly named and variables in local scope can
be modified by lambda capture alleviating the need for a return value. 


\section{Conclusion}

We have presented a method for defining inline visitors in standard C++. The
method does not, by design, remove the tight coupling between visited and
visiting class heirarchies. We have found the tight coupling to be desirable
when the addition of new types necessitates their explicit consideration by
relevant domain experts. 

Performance, portability and convenience of the inline visitor mean that we
would encourage its use where tight-coupling is acceptable and type-specific
handling is logically localized.

\end{document}

%[REF: Gang of four] : E.Gamma et al. Design Patterns ( Addison-Wesley Longman, 1995) 
%[REF: Scott] : http://www.artima.com/cppsource/top_cpp_aha_moments.html - "Visitor lets you define a new operation without changing the classes of the elements on which it operates."
%[REF: Acyclic Visitor, Loki] : http://www.objectmentor.com/resources/articles/acv.pdf; http://loki-lib.sourceforge.net/
%[REF: Cooperative visitor] http://www.artima.com/cppsource/cooperative_visitor.html

% end of file %
